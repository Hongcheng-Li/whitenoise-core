\documentclass[11pt]{scrartcl} % Font size
%%%%%%%%%%%%%%%%%%%%%%%%%%%%%%%%%%%%%%%%%
% Wenneker Assignment
% Structure Specification File
% Version 2.0 (12/1/2019)
%
% This template originates from:
% http://www.LaTeXTemplates.com
%
% Authors:
% Vel (vel@LaTeXTemplates.com)
% Frits Wenneker
%
% License:
% CC BY-NC-SA 3.0 (http://creativecommons.org/licenses/by-nc-sa/3.0/)
%
%%%%%%%%%%%%%%%%%%%%%%%%%%%%%%%%%%%%%%%%%

%----------------------------------------------------------------------------------------
%	PACKAGES AND OTHER DOCUMENT CONFIGURATIONS
%----------------------------------------------------------------------------------------

\usepackage{amsmath, amsfonts, amsthm} % Math packages
\usepackage{hyperref}
\usepackage{comment}
\usepackage{amsmath}
\usepackage{amssymb}
\usepackage{bbm}
\usepackage{algpseudocode, algorithm, algorithmicx}
\usepackage[toc,page]{appendix}
\usepackage[user,titleref]{zref}

\hypersetup{
   colorlinks=true,
   citecolor=red,
   linkcolor=cyan,
   filecolor=magenta,
   urlcolor=blue,
 }

\usepackage{url}
\usepackage{listings} % Code listings, with syntax highlighting

\usepackage[english]{babel} % English language hyphenation

\usepackage{graphicx} % Required for inserting images
\graphicspath{{Figures/}{./}} % Specifies where to look for included images (trailing slash required)

\usepackage{booktabs} % Required for better horizontal rules in tables

\numberwithin{equation}{section} % Number equations within sections (i.e. 1.1, 1.2, 2.1, 2.2 instead of 1, 2, 3, 4)
\numberwithin{figure}{section} % Number figures within sections (i.e. 1.1, 1.2, 2.1, 2.2 instead of 1, 2, 3, 4)
\numberwithin{table}{section} % Number tables within sections (i.e. 1.1, 1.2, 2.1, 2.2 instead of 1, 2, 3, 4)

\setlength\parindent{0pt} % Removes all indentation from paragraphs

\usepackage{enumitem} % Required for list customisation
\setlist{noitemsep} % No spacing between list items

%----------------------------------------------------------------------------------------
%	DOCUMENT MARGINS
%----------------------------------------------------------------------------------------

\usepackage{geometry} % Required for adjusting page dimensions and margins

\geometry{
	paper=a4paper, % Paper size, change to letterpaper for US letter size
	top=2.5cm, % Top margin
	bottom=3cm, % Bottom margin
	left=3cm, % Left margin
	right=3cm, % Right margin
	headheight=0.75cm, % Header height
	footskip=1.5cm, % Space from the bottom margin to the baseline of the footer
	headsep=0.75cm, % Space from the top margin to the baseline of the header
	%showframe, % Uncomment to show how the type block is set on the page
}

%----------------------------------------------------------------------------------------
%	FONTS
%----------------------------------------------------------------------------------------

\usepackage[utf8]{inputenc} % Required for inputting international characters
\usepackage[T1]{fontenc} % Use 8-bit encoding

% \usepackage{fourier} % Use the Adobe Utopia font for the document

%----------------------------------------------------------------------------------------
%	SECTION TITLES
%----------------------------------------------------------------------------------------

\usepackage{sectsty} % Allows customising section commands

\sectionfont{\vspace{6pt}\centering\normalfont\scshape} % \section{} styling
\subsectionfont{\normalfont\bfseries} % \subsection{} styling
\subsubsectionfont{\normalfont\bfseries} % \subsubsection{} styling
\paragraphfont{\normalfont\scshape} % \paragraph{} styling

%----------------------------------------------------------------------------------------
%	HEADERS AND FOOTERS
%----------------------------------------------------------------------------------------

\usepackage{scrlayer-scrpage} % Required for customising headers and footers

\ohead*{} % Right header
\ihead*{} % Left header
\chead*{} % Centre header

\ofoot*{} % Right footer
\ifoot*{} % Left footer
\cfoot*{\pagemark} % Centre footer

%-----------------------------------------------------------
%  Define new commands
%-----------------------------------------------------------
\DeclareMathOperator*{\argmax}{arg\,max}
\DeclareMathOperator*{\argmin}{arg\,min}
\DeclareMathOperator\supp{supp}

\newcommand*\Let[2]{\State #1 $\gets$ #2}

\DeclareMathOperator*{\E}{\mathbbm{E}}
\DeclareMathOperator*{\Z}{\mathbbm{Z}}
\DeclareMathOperator*{\R}{\mathbbm{R}}

\newtheorem{theorem}{Theorem}


\newcommand{\AuthorNote}[2]{{\color{red} {\textbf [[ #1:} #2 {\textbf ]]}}}
\newcommand{\cnote}[1]{\AuthorNote{Christian}{#1}} % Include the file specifying the document structure and custom commands

%----------------------------------------------------------------------------------------
%	TITLE SECTION
%----------------------------------------------------------------------------------------

\title{
	\normalfont\normalsize
	\textsc{Harvard Privacy Tools Project}\\ % Your university, school and/or department name(s)
	\vspace{25pt} % Whitespace
	\rule{\linewidth}{0.5pt}\\ % Thin top horizontal rule
	\vspace{20pt} % Whitespace
	{\huge Data Processing Notes}\\ % The assignment title
	\vspace{12pt} % Whitespace
	\rule{\linewidth}{2pt}\\ % Thick bottom horizontal rule
	\vspace{12pt} % Whitespace
}

\author{} % Your name

\date{} % Today's date (\today) or a custom date

\begin{document}

\maketitle

\tableofcontents

\section{Overview}
This document contains notes on our choices regarding ``data processing'', which I am using to stand in
for all things related to data bookkeeping (data bounds, imputation, sample size calculations, etc.).

\section{Clamping}

\subsection{Introduction}
Many operations in the library require the definition of a set of possible data values
we would like to use for said operation. For numeric variables, this typically
takes the form of a closed interval $[min, max] \in \R$ or $\Z$. For categorical
variables, this is a discrete set of elements. The \emph{clamp} component sets these 
properties for a given analysis.
In our system, we have slightly different notions of clamping for numeric and categorical variables.
We consider $f64$ and $i64$ to be allowable numeric types, and $i64, bool,$ and $String$ to be
allowable categorical types. \newline

Our method for clamping numeric types should be familiar; it requires data, a lower bound, and an upper bound,
and maps data elements outside of the interval $[min, max]$ to the nearest point in the interval.
Null values (represented as NAN for the $f64$ type and which do not exist for the $i64$ type)
are preserved by numeric clamping. \newline

Our categorical clamping requires data, a set of feasible values, and a null value.
Data elements that are outside the set of feasible values are mapped to the null value.
Null values are preserved by the categorical clamping. \newline 

\subsection{Algorithm Statement}
The component arguments are as follows:
\begin{enumerate}
    \item X: The data of size $n \times m$ to be clamped.
    \item categories: A vector of length $m$, where each element is the set of plausible data values 
                      for column $m$ of $X$. If not provided, defaults to None.
    \item null\_value: A vector of length $m$, where each element is the designated ``null'' 
                      for column $m$ of $X$. That is, elements $\not\in$ categories are mapped to the 
                      null\_value. If not provided, defaults to None.
    \item lower: A vector of length $m$, where each element is a proposed minimum value for column $m$ of $X$.
                 If not provided, defaults to None.
    \item upper: A vector of length $m$, where each element is a proposed maximum value for column $m$ of $X$.
                 If not provided, defaults to None. 
\end{enumerate}

\begin{algorithm}[H]
    \label{alg:clamping}
    \caption{Clamping: clamp($X$, categories, lower, upper)}
    \begin{algorithmic}[1]
        \For{$j \in [m]$} \Comment{Iterate over columns}
            \For{$i \in [n]$} \Comment{Iterate over rows}
                \If{categories $\neq$ None} \Comment{Proceed with categorical clamping}
                    \If{$X_{i,j} \not\in categories[j]$}
                        \State $X_{i,j} \gets null\_value[j]$
                    \EndIf
                \Else \Comment{Proceed with numeric clamping}
                    \If{$X_{i,j} < lower[j]$}
                        \State $X_{i,j} \gets lower[j]$
                    \ElsIf{$X_{i,j} > upper[j]$}
                        \State $X_{i,j} \gets upper[j]$
                    \EndIf
                \EndIf
            \EndFor
        \EndFor
        \\ \Return $X$
    \end{algorithmic}
\end{algorithm}

\section{Imputation} 
\label{sec:impute}

\subsection{Introduction}
Some operations in the library require the private data to be non-null. The \emph{impute} component 
sets the non-null property for an analysis. \newline 

Much like clamping, we have slightly different notions of imputation for numeric and categorical variables.
In this case, we consider only $f64$ to be an allowable numeric type, while
$f64, i64, bool,$ and $String$ are allowable categorical types. We do not allow for numeric imputation on $i64$
because $i64$ does not support NAN values and thus are non-null by default. \newline

Numeric imputation is parameterized by min, max, and a supported probability distribution (Gaussian or Uniform)
with appropriate arguments. NAN values are replaced with a draw from the provided distribution. \newline

Categorical imputation is parameterized by categories, probabilities, and a null value.
Each data element equal to the null value is replaced with a draw from the set of categories,
according to the probabilities provided.

\subsection{Algorithm Statement}
The component arguments are as follows:
\begin{enumerate}
    \item X: The data of size $n \times m$ for which values should be imputed.
    \item categories: A vector of length $m$, where each element is the set of plausible data values 
                      for column $m$ of $X$. If not provided, defaults to None.
    \item weights: A vector of length $m$, where each element is a set of imputation weights associated with the categories 
                   for column $m$ of $X$. If not provided, defaults to None.
    \item null\_value: A vector of length $m$, where each element is the designated ``null'' 
                      for column $m$ of $X$. If an element is equal to the corresponding null\_value
                      for its column, it is imputed from the probability distribution over the categories
                      defined by the weights.
    \item distribution: A string providing the distribution from which elements will be imputed. 
                        Currently supported options are ``Uniform'' and ``Gaussian''. 
                        Used only if \emph{categories} == None. 
                        Defaults to ``Uniform''.
    \item shift: A vector of length $m$, where each element is the expectation of the Gaussian distribution used 
                 for imputing missing values in column $m$ of $X$. Used only if \emph{distribution} == ``Gaussian''.
                 Defaults to None.
    \item scale: A vector of length $m$, where each element is the standard deviation of the Gaussian distribution used
                 for imputing missing values in column $m$ of $X$. Used only if \emph{distribution} == ``Gaussian''.
                 Defaults to None. 
    \item lower: A vector of length $m$, where each element is a proposed minimum value for column $m$ of $X$.
                 Acts as a lower bound for the distribution from which elements are imputed.
                 If not provided, defaults to None.
    \item upper: A vector of length $m$, where each element is a proposed maximum value for column $m$ of $X$.
                 Acts as a lower bound for the distribution from which elements are imputed.
                 If not provided, defaults to None. 
\end{enumerate}

Let Categorical$(k, (p_1, \hdots, p_k))$ be the \href{https://en.wikipedia.org/wiki/Categorical_distribution}{categorical distribution}, which chooses 
a single element from the set $\{1, 2, \hdots, k\}$ where each element $i$ has probability of being chosen $p_i$.
Then we have
\begin{algorithm}[H]
    \label{alg:imputation}
    \caption{Imputation: \newline impute($X$, categories, weights, null\_value, distribution, shift, scale, lower, upper)}
    \begin{algorithmic}[1]
        \For{$j \in [m]$} \Comment{Iterate over columns}
            \For{$i \in [n]$} \Comment{Iterate over rows}
                \If{categories $\neq$ None} \Comment{Proceed with categorical imputation}
                    \If{$X_{i,j} == null\_value[j]$}
                        \State sampling\_index $\gets$ Categorical$(\big\vert \text{weights}[j] \big\vert, \text{weights}[j])$ 
                        \State $X_{i,j} \gets$ categories[j][sampling\_index] 
                    \EndIf
                \Else \Comment{Proceed with numeric imputation}
                    \If{$X_{i,j} ==$ NAN}
                        \If{distribution $==$ ``Uniform''}
                            \State $X_{i,j} \gets$ Uniform(lower[j], upper[j])
                        \ElsIf{distribution $==$ ``Gaussian''}
                            \State $X_{i,j} \gets$ Gaussian(shift[j], scale[j])
                            \If{$X_{i,j} < lower[j]$}
                                \State $X_{i,j} \gets lower[j]$
                            \ElsIf{$X_{i,j} > upper[j]$}
                                \State $X_{i,j} \gets upper[j]$
                            \EndIf
                        \EndIf
                    \EndIf
                \EndIf
            \EndFor
        \EndFor
        \\ \Return $X$
    \end{algorithmic}
\end{algorithm}

\section{Resize}
\label{sec:resize}

\subsection{Motivation}
We want our library to support cases both in which the analyst does and does not have access to
the number of records in the data. Doing the latter well is an under-explored problem, so we 
propose the \emph{resize} framework as a possible solution.

\subsection{Goals}
The \emph{resize} framework is a means of jointly achieving a few different goals:
\begin{enumerate}
    \item guarantee known $n$ for analyses that require it,
    \item give users flexibility in how they trade off between bias and variance, and
    \item allow for both c-stability and privacy amplification from subsampling within the same privacy calculus.
\end{enumerate}

\subsection{Algorithm Statement}
\label{subsec:algorithm_statement}
The \emph{resize} function takes the following inputs:
\begin{enumerate}
    \item $X$: The private underlying data.
    \item $\tilde{n}$: The size of the private underlying data.
    \item $n$: The desired size of the new data.
    \item $p$: The proportion of the underlying data that can be used to construct the new data. $p$ can be $> 1$.
    \item $...$: Various arguments explaining imputation rules (these will follow the rules from section~\ref{sec:impute}).
\end{enumerate}

Let $sample(Y, m)$ be a function that samples $m$ elements from data set $Y$ without replacement. 
Let $Aug(Y, m, \hdots)$ be a function that imputes new elements independent of the data (using imputation parameters given by $\hdots$) for a data set $Y$ until it is of size $m$. 
The algorithm will look something like the following:
\begin{algorithm}[H]
    \caption{Resize: resize($X$, $n$, $p$, neighboring, ...)}
    \label{alg:gen_resize}
    \begin{algorithmic}[1]
        \State $c \gets \lceil p \rceil$ \Comment{sets c-stability property}
        \State $s \gets p/c$ \Comment{sets subsampled\_proportion property}
        \State $X_c \gets \bigcup_{i=1}^{c}X$ \Comment{create new database of size $c\tilde{n}$, composed of $c$ copies of $X$}
        \If{neighboring == ``replace one''}
            \State $m \gets \lfloor sc\tilde{n} \rfloor$ \Comment{number of records that can be filled using subsampled private data}
        \ElsIf{neighboring == ``add/remove one''}
            \State $m \gets Binomial(c\tilde{n}, s)$ \Comment{number of records that can be filled using subsampled private data}
        \EndIf
        \State $(\epsilon', \delta') \gets \left(\log\left(1+s\left(e^{c\epsilon}-1\right) \right), s\left(\sum_{i=0}^{c-1}e^{i \epsilon}\right)\delta \right)$ \Comment{privacy amplification via subsampling}
        \State $X' \gets sample\left( X_c, \max(m, n) \right) \bigcup \left( Aug(\varnothing, \max(0, n - m), \hdots) \right)$
        \\ \Return $(X', \epsilon', \delta')$
    \end{algorithmic}
\end{algorithm}

The $\epsilon', \delta'$ terms come from first applying the group privacy definition with group size $c$ to the
database $X_c$ to get $\left(c\epsilon, \left(\sum_{i=0}^{c-1}e^{i \epsilon}\right)\delta \right)$ \cite{Vad17}
and then applying privacy amplification by subsampling results from Theorems 8 and 9 of \cite{BBG18}.
Note that, for the ``replace one'' definition, we could be using $\frac{m}{c\tilde{n}}$ instead of $s$ in the privacy calculation. Using $s$ gives us a very slightly 
worse privacy guarantee (the only difference is the $\lfloor \cdot \rfloor$ we used to get $m$), but is nice for 
consistency between the methods and not having to keep track of $m$ as an extra property.

\subsection{Functional Privacy Parameters}
We established in section~\ref{subsec:algorithm_statement} that a user asking for an $(\epsilon, \delta)$-DP guarantee will get an 
$(\epsilon', \delta')$-DP guarantee with respect to the original private data. What we'd really like, however, is for the user to ask for an $(\epsilon, \delta)$-DP
guarantee and have the library come up with what we will call a \emph{functional $(\epsilon_f, \delta_f)$} that will ensure $(\epsilon, \delta)$-DP
on the original data. Any components that operate on the resized data will use the \emph{functional $(\epsilon_f, \delta_f)$} internally instead of
the parameters passed by the user.

\begin{theorem}
    A mechanism that respects
    \[ (\epsilon_f, \delta_f) = \left( \frac{1}{c}\log\left(\frac{e^{\epsilon}-1}{s} + 1 \right), \frac{\delta}{s\left(\sum_{i=0}^{c-1}e^{i \epsilon}\right)} \right)\text{-DP} \] 
    on the resized data respects $(\epsilon, \delta)$-DP on the true private data.
    \begin{proof}
        We know that an $(\epsilon, \delta)$-DP on the resized data corresponds to a 
        \[ (\epsilon', \delta') = \left(\log\left(1+s\left(e^{c\epsilon}-1\right) \right), s\left(\sum_{i=0}^{c-1}e^{i \epsilon}\right)\delta \right) \]
        guarantee on the original data. But we want our mechanism to respect $(\epsilon_f, \delta_f)$ on the resized data such that it 
        respects $(\epsilon, \delta)$ on the original data. So we replace $(\epsilon, \delta)$ with 
        $\epsilon_f, \delta_f$ and $(\epsilon', \delta')$ with $(\epsilon, \delta)$. 
        Then we can invert the function to get what we want. \newline 

        Let's start with $\epsilon, \epsilon_f$:
        \begin{align*}
            \epsilon &= \log\left(1+s\left(e^{c\epsilon_f}-1\right)\right) \\
            e^{\epsilon} &= 1+s\left(e^{c\epsilon_f}-1\right) \\
            \frac{e^{\epsilon}-1}{s} &= e^{c\epsilon_f}-1 \\
            \frac{1}{c}\log\left(\frac{e^{\epsilon}-1}{s} + 1 \right) &= \epsilon_f.
        \end{align*}

        We carry out a similar calculation for $\delta, \delta_f$:
        \begin{align*}
            \delta &= s\left(\sum_{i=0}^{c-1}e^{i \epsilon}\right)\delta_f \\
            \frac{\delta}{s\left(\sum_{i=0}^{c-1}e^{i \epsilon}\right)} &= \delta_f.
        \end{align*}
    \end{proof} 
    So, in order for a mechanism to respect $(\epsilon, \delta)$-DP on the original data, it must respect 
    $\left( \frac{1}{c}\log\left(\frac{e^{\epsilon}-1}{s} + 1 \right), \frac{\delta}{s\left(\sum_{i=0}^{c-1}e^{i \epsilon}\right)} \right)$-DP
    on the resized data.
\end{theorem}

We now present an mini-algorithm for finding the \emph{functional} $(\epsilon, \delta)$.
\begin{algorithm}
    \caption{Finding Functional $(\epsilon_f, \delta_f)$: get\_func\_priv($p, \epsilon, \delta$)}
    \label{alg:finding_functional_privacy}
    \begin{algorithmic}[1]
        \State $c \gets \lceil p \rceil$ \Comment{sets c-stability property}
        \State $s \gets p/c$ \Comment{sets subsampled\_proportion property}
        \State $\epsilon_{f} \gets \frac{1}{c}\log\left(\frac{e^{\epsilon}-1}{s} + 1\right)$
        \State $\delta_{f} \gets \frac{\delta}{s\left(\sum_{i=0}^{c-1}e^{i \epsilon}\right)}$
        \\ \Return $(\epsilon_{f}, \delta_{f})$ 
    \end{algorithmic}
\end{algorithm}

\subsection{Examples} \label{subsec:examples}
Let $X$ be such that $\tilde{n} = \vert X \vert = 100$. We can look at a few examples of calls to the resize function (which will return $X'$) and check the behavior.
\begin{enumerate}
    \item resize(X, 150, 1, ...): $X'$ will be made up of the 100 true elements of $X$ and 50 imputed values. The functional privacy parameters are identical to the ones the user provides. 
    \item resize(X, 100, 0.75, ...): $X'$ will be made up of 75 true elements of $X$ and 25 imputed values. The functional privacy parameters will benefit (lower noise) from amplification via subsampling.
    \item resize(X, 90, 1.5, ...): $X'$ will be a random sample of $X \bigcup X$ of size 90. 
            The functional privacy parameters will lead to greater noise than what the user provides, as they have to take into account the new c-stability of $2$. 
            This example is illustrative in that is shows that the functional privacy usage is affected only by $p$ -- it has nothing to do with the relative sizes of the $X$ and $X'$.
\end{enumerate}

\subsection{How to choose $n$} \label{subsec:choosing_n}
To this point, we have ignored how a user would go about choosing $n$. If no trusted public metadata exist, we 
suggest estimating the size of the private database (e.g. using the Geometric mechanism) and using that release 
for $n$.  

\subsection{Accuracy/Error}
The accuracy/error calculations in the library are always relative to the resized data $X'$, rather than the true private data $X$. 
We are currently thinking about how best to extend guarantees from $X'$ to $X$.

\bibliographystyle{alpha}
\bibliography{data_processing}
\end{document}