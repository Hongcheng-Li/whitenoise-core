\documentclass[11pt]{scrartcl} % Font size
%%%%%%%%%%%%%%%%%%%%%%%%%%%%%%%%%%%%%%%%%
% Wenneker Assignment
% Structure Specification File
% Version 2.0 (12/1/2019)
%
% This template originates from:
% http://www.LaTeXTemplates.com
%
% Authors:
% Vel (vel@LaTeXTemplates.com)
% Frits Wenneker
%
% License:
% CC BY-NC-SA 3.0 (http://creativecommons.org/licenses/by-nc-sa/3.0/)
%
%%%%%%%%%%%%%%%%%%%%%%%%%%%%%%%%%%%%%%%%%

%----------------------------------------------------------------------------------------
%	PACKAGES AND OTHER DOCUMENT CONFIGURATIONS
%----------------------------------------------------------------------------------------

\usepackage{amsmath, amsfonts, amsthm} % Math packages
\usepackage{hyperref}
\usepackage{comment}
\usepackage{amsmath}
\usepackage{amssymb}
\usepackage{bbm}
\usepackage{algpseudocode, algorithm, algorithmicx}
\usepackage[toc,page]{appendix}
\usepackage[user,titleref]{zref}

\hypersetup{
   colorlinks=true,
   citecolor=red,
   linkcolor=cyan,
   filecolor=magenta,
   urlcolor=blue,
 }

\usepackage{url}
\usepackage{listings} % Code listings, with syntax highlighting

\usepackage[english]{babel} % English language hyphenation

\usepackage{graphicx} % Required for inserting images
\graphicspath{{Figures/}{./}} % Specifies where to look for included images (trailing slash required)

\usepackage{booktabs} % Required for better horizontal rules in tables

\numberwithin{equation}{section} % Number equations within sections (i.e. 1.1, 1.2, 2.1, 2.2 instead of 1, 2, 3, 4)
\numberwithin{figure}{section} % Number figures within sections (i.e. 1.1, 1.2, 2.1, 2.2 instead of 1, 2, 3, 4)
\numberwithin{table}{section} % Number tables within sections (i.e. 1.1, 1.2, 2.1, 2.2 instead of 1, 2, 3, 4)

\setlength\parindent{0pt} % Removes all indentation from paragraphs

\usepackage{enumitem} % Required for list customisation
\setlist{noitemsep} % No spacing between list items

%----------------------------------------------------------------------------------------
%	DOCUMENT MARGINS
%----------------------------------------------------------------------------------------

\usepackage{geometry} % Required for adjusting page dimensions and margins

\geometry{
	paper=a4paper, % Paper size, change to letterpaper for US letter size
	top=2.5cm, % Top margin
	bottom=3cm, % Bottom margin
	left=3cm, % Left margin
	right=3cm, % Right margin
	headheight=0.75cm, % Header height
	footskip=1.5cm, % Space from the bottom margin to the baseline of the footer
	headsep=0.75cm, % Space from the top margin to the baseline of the header
	%showframe, % Uncomment to show how the type block is set on the page
}

%----------------------------------------------------------------------------------------
%	FONTS
%----------------------------------------------------------------------------------------

\usepackage[utf8]{inputenc} % Required for inputting international characters
\usepackage[T1]{fontenc} % Use 8-bit encoding

% \usepackage{fourier} % Use the Adobe Utopia font for the document

%----------------------------------------------------------------------------------------
%	SECTION TITLES
%----------------------------------------------------------------------------------------

\usepackage{sectsty} % Allows customising section commands

\sectionfont{\vspace{6pt}\centering\normalfont\scshape} % \section{} styling
\subsectionfont{\normalfont\bfseries} % \subsection{} styling
\subsubsectionfont{\normalfont\bfseries} % \subsubsection{} styling
\paragraphfont{\normalfont\scshape} % \paragraph{} styling

%----------------------------------------------------------------------------------------
%	HEADERS AND FOOTERS
%----------------------------------------------------------------------------------------

\usepackage{scrlayer-scrpage} % Required for customising headers and footers

\ohead*{} % Right header
\ihead*{} % Left header
\chead*{} % Centre header

\ofoot*{} % Right footer
\ifoot*{} % Left footer
\cfoot*{\pagemark} % Centre footer

%-----------------------------------------------------------
%  Define new commands
%-----------------------------------------------------------
\DeclareMathOperator*{\argmax}{arg\,max}
\DeclareMathOperator*{\argmin}{arg\,min}
\DeclareMathOperator\supp{supp}

\newcommand*\Let[2]{\State #1 $\gets$ #2}

\DeclareMathOperator*{\E}{\mathbbm{E}}
\DeclareMathOperator*{\Z}{\mathbbm{Z}}
\DeclareMathOperator*{\R}{\mathbbm{R}}

\newtheorem{theorem}{Theorem}


\newcommand{\AuthorNote}[2]{{\color{red} {\textbf [[ #1:} #2 {\textbf ]]}}}
\newcommand{\cnote}[1]{\AuthorNote{Christian}{#1}} % Include the file specifying the document structure and custom commands

%----------------------------------------------------------------------------------------
%	TITLE SECTION
%----------------------------------------------------------------------------------------

\title{
	\normalfont\normalsize
	\textsc{Harvard Privacy Tools Project}\\ % Your university, school and/or department name(s)
	\vspace{25pt} % Whitespace
	\rule{\linewidth}{0.5pt}\\ % Thin top horizontal rule
	\vspace{20pt} % Whitespace
	{\huge The Exponential Mechanism for Medians}\\ % The assignment title
	\vspace{12pt} % Whitespace
	\rule{\linewidth}{2pt}\\ % Thick bottom horizontal rule
	\vspace{12pt} % Whitespace
}

% \author{\LARGE} % Your name

\date{\normalsize\today} % Today's date (\today) or a custom date

\begin{document}
\maketitle

\section{The Exponential Mechanism}

Sometimes, the global sensitivity of a function is too great, so the Laplace mechanism will not produce meaningful results. The median is one such function. In many cases, the \textit{Exponential mechanism} is an alternate approach that gives reasonable utility.\footnote{This is not the \textit{only} advantage of the exponential mechanism. It is a way to compute differentially private queries on non-numeric data, unlike the Laplace mechanism it does not assume that the probability of outputting a response ought to be symmetric about the true response, etc.} Introduced in 2007 by McSherry and Talwar, the exponential mechanism posits that for a given database, users prefer some outputs over others. That those preferences may be encapsulated with a utility score, where a high utility score indicates a higher preference for that output. The exponential mechanism releases outputs with probability proportional (in the exponent) to the utility score and the sensitivity of the utility function. 

\begin{definition}  
Let $\mathcal{X}$ be a space of databases and let $[m,M]$ be an arbitrary range. Let $u: \mathcal{X} \times [m,M] \rightarrow \mathbb{R}$ be a utility function, which maps pairs of databases and outputs to a utility score. Let $\Delta u$ be the sensitivity of $u$ with respect to the database argument. The exponential mechanism outputs $r \in [m,M]$ with probability proportional to $\exp\left(\frac{\varepsilon u(x,r)}{2 \Delta u}\right)$ \cite{mcsherry2007mechanism, dwork2014algorithmic}.\footnote{The original definition is from \cite{mcsherry2007mechanism}, but here we state the version rewritten in \cite{dwork2014algorithmic} as it is slightly clearer.}
\end{definition}

\begin{theorem}
The exponential mechanism preserves $(\varepsilon,0)$-differential privacy \cite{mcsherry2007mechanism, dwork2014algorithmic}.\footnote{As written in \cite{mcsherry2007mechanism}, the mechanism actually preserves $(2\varepsilon\Delta u,0)$-differential privacy; the main difference in the $\cite{dwork2014algorithmic}$ version is that it has the extra factor of $2\Delta u$ to avoid these extra terms.}
\end{theorem}

Note that the exponential mechanism may not be tractable in many cases, as it assumes the existence of a utility function, and even if one exists it may not be tractable to compute it efficiently. 

\section{An Exponential Mechanism for a quantile}

\subsection{Defining a sensible utility function}

Note that a user will prefer an output that is closer to the true quantile over one that is further away. Let $x$ be an (ordered) data set, let $r$ be a possible output, and let $N$ be the size of the data set. Let $\#(Z>r)$ refer to the number of points in $x$ above $r$. Then, the following is a reasonable utility function for a release $r$ for the $\alpha$-quantile of $x$.

\begin{equation}
u(x,r) = \max(\alpha, (1-\alpha))N - \vert (1-\alpha)\#(Z<r) - \alpha\#(Z>r)\vert.
\end{equation} 

\subsection{Sensitivity of the utility function}

\subsubsection{Neighboring Definition: Change One}
\begin{lemma}
The above utility function $u$ has $\ell_11$ sensitivity bounded above by 1 in the change one model.
\end{lemma}

\begin{proof}
Let $c_1 = \#(Z<r)$ and $c_2 = \#(Z>r)$. In one worst case, $c_1$ increases by 1 and $c_2$ decreases by 1.
Then, 
\begin{align*}
\Delta u &= \vert (1-\alpha) (c_1 + 1) - \alpha (c_2-1) \vert - \vert (1-\alpha) c_1 - \alpha c_2 \vert  \\
 &\le \vert (1-\alpha) (c_1 + 1) - \alpha (c_2-1) - (1-\alpha) c_1 + \alpha c_2 \vert\\
 & \le \vert c_1 + 1 - \alpha c_1 - \alpha - \alpha c_2 + \alpha - c_1 + \alpha c_1 + \alpha c_2 \vert\\
&= 1
\end{align*}
If instead $c_2$ decreases by 1 and $c_1$ increases by 1, the same thing will happen except with a negative sign that will not impact the final result due to the absolute values.
\end{proof}
\subsubsection{Neighboring Definition: Add/Drop One}

\begin{lemma}
The above utility function $u$ has $\ell_11$ sensitivity bounded above by $\max(1-\alpha, \alpha)$ in the add/drop one model.
\end{lemma}

\begin{proof}
Let $c_1 = \#(Z<r)$ and $c_2 = \#(Z>r)$.  Consider what happens if one point is added. There are two cases that would impact the utility function: 
\begin{enumerate}
\item $c_1$ increases by one and nothing happens to $c_2$.
\item $c_2$ increases by one and nothing happens to $c_1$.
\end{enumerate}

Say the first case occurs. Then,

\begin{align*}
\Delta u &= | (1-\alpha) (c_1 + 1) - \alpha (c_2) | - | (1-\alpha) c_1 - \alpha c_2 |  \\
 	&\le | (1-\alpha) (c_1 + 1) - \alpha (c_2) - (1-\alpha) c_1 + \alpha c_2 | \\
	&= 1 - \alpha
\end{align*}

In the second case,

\begin{align*}
\delta u &= \vert (1-\alpha) (c_1) - \alpha (c_2 + 1) \vert - \vert (1-\alpha) c_1 - \alpha c_2 \vert \\
	&\le \vert c_1 -\alpha c_1 - \alpha c_2 - \alpha - c_1 + \alpha c_1 + \alpha c_2  \vert\\
	&= \alpha 
\end{align*}
\end{proof}

Subtracting a point leads to the same results. 

\bibliographystyle{alpha}
\nocite{*}
\bibliography{expMechMedian}
\end{document}